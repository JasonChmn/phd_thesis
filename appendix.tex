\appendix

\chapter{Reinforcement Learning: Overview}
\label{appendix:rl_overview}

Reinforcement Learning can be defined as an agent interacting with an environment according to a policy $\pi$. 
The interaction sequence can be modeled as a Markov Decision Process (MDP) that is a tuple ($S$,$A$,$P$,$R$,$\mu$,$\gamma$), where:
\begin{itemize}
    \item $S$ is the state space of the environment.
    \item $A$ the set of discrete or continuous actions available.
    \item $P: S \times A \rightarrow \Delta S$ is the transition function of the MDP dynamics.
    \item $R: S \times A \rightarrow \mathbb{R}$ is the reward function defining the desired agent behavior.
    \item $\mu$ is the initial state distribution.
    \item $\gamma \in [0,1]$ a discount factor.
\end{itemize}

As the full environment state is often not observable by our agent, we can use partially observable MDP like in this thesis, where our steering method only uses local observations to navigate through the environment.

The agent starts an episode in an initial state $s_0$ sampled according to $\mu$.
Depending on the agent state $s_t \in S$, the objective of the policy $\pi$ is to make the agent perform an action $a=\pi(s)$ in its environment, while maximizing its future cumulative rewards $J(\pi) = \mathbb{E}[\sum_{k=t}^{\infty} \gamma^k r_{t+k} | \pi]$.
We denote the optimal policy $\pi^* = \underset{\pi}{\mbox{arg max}} \; J(\pi)$.





\chapter{Mixed-Integer Programming Formulation details}

\section{Feasibility Constraints.\label{appendix:feasibility_constr}}
We denote $\mathcal{F}$ the set of kinematic and dynamic feasibility constraints, as explained in \cite{sl1m_v1}. 
It includes some constraints on the robot center of mass for its equilibrium and the reachability of the planned contacts.
As a result, they guarantee the feasibility of the robot contacts, which are characterized by their position $p$ and orientation $r$.

\paragraph{Center of mass constraints}
\begin{figure}[t]
    \centering
    \captionsetup[subfigure]{justification=centering}
    \includegraphics[width=0.8\textwidth]{Figures/Appendix/2pac_feasibility.png}
    \caption{Computation of a feasible center of mass quasi-static trajectory using the 2PAC method \cite{Tonneau2018_2PAC}. Source: Tonneau et al. \cite{sl1m_v1}}
    \label{fig:mip:feasibility_constr}
\end{figure}
We guarantee the equilibrium and balance constraints using the 2PAC formulation \cite{Tonneau2018_2PAC}.
These constraints will be succintly explained and we refer the reader to \cite{sl1m_v1} for further details.
Using this formulation, we only need to select 2 Center Of Mass (COM) positions for each phase $k$ from step $\mbox{p}_{k-1}$ to $\mbox{p}_k$, that are $\mbox{c}_{k,\:0}$ and $\mbox{c}_{k,\:1}$ to guarantee continuous feasibility (Figure \ref{fig:mip:feasibility_constr}).

In the context of biped walking, a sufficient condition for static equilibrium is to ensure that the center of mass lies above the support effector.
The constraint can be formulated as follows:
\begin{equation}
    \label{eq:equilibrium_constr}
    \begin{aligned}
        F_{k-1} (\mbox{c}_{k,\:0}-\mbox{p}_{k-1}) &\leq  f_{k-1}\\% + M a_{k-1}^{j}\\
        F_{k} (\mbox{c}_{k,\:1}-\mbox{p}_{k}) &\leq  f_{k}% + M a{k}^{j}
    \end{aligned}
\end{equation}
where $F_k$ and $f_k$ are the matrix and vector defining the foot polygonal shape at position $\mbox{p}_k$ (considering the contact lies on flat ground).
Constraints (\ref{eq:equilibrium_constr}) depends only on the xy coordinates of the COM.
As a result, by convexity of the static equilibrium regions, the straight lines $[\mbox{c}_{k,\:0}, \mbox{c}_{k,\:1}]$ continuously satisfies the static equilibrium
constraint, as well as $[\mbox{c}_{k-1,\:1}, \mbox{c}_{k,\:0}]$ and $[\mbox{c}_{k,\:1}, \mbox{c}_{k+1,\:0}]$ as the COM stays above the corresponding support effector.

\paragraph{Reachability constraints.}
We also use the center of mass positions $\mbox{c}_{k,\:0}$ and $\mbox{c}_{k,\:1}$ to guarantee kinematic reachability.
A 3D polytope $\mathcal{R}$ is obtained for each effector (feet in our case) via offline random sampling, approximating the reachable COM workspace.
The resulting polytope is expressed as follows: $\mathcal{R} : \{\mbox{c} \in \mathbb{R}^3 , \; \! R \mbox{c} \; \leq \; \! r\}$, where $R$ and $r$ are the matrix and vector defining the polytope.

For each phase $k$, we consider the orientation of the foot frame constant and equal to $r_{k-1}$.
We note $\mathcal{R}_k$ the rotated polytope associated with contact $\mbox{p}_{k}$.
For phase $k$, the contraints on the COM positions $\mbox{c}_{k,\:0}$ and $\mbox{c}_{k,\:1}$ can be formulated as follows:
\begin{equation}
    %R_{l}^j (\mbox{c}_{l,e} - \mbox{p}_{l}) \leq r_l^j + M a_l^j \;\;\;\; \forall j, \forall l \in \{k-1,k\}
    R_{l} (\mbox{c}_{l,e} - \mbox{p}_{l}) \leq r_l \;\;\;\; \forall l \in \{k-1,k\}
\end{equation}

\paragraph{Relative foot position constraints.\label{appendix:foot_pos_constr}}
Just as for the COM reachability, a 3D polytope $\mathcal{Q}$ is randomly sampled offline (or manually given) to approximate the reachable workspace of each foot with respect to the other. The polytope rotated by $r_{k-1}$ then translated by $\mbox{p}_{k-1}$ can be expressed as $\mathcal{Q}_k: \{ \mbox{p} \in \mathbb{R}^3, Q_k \mbox{p} \leq q_k \}$, where $Q_k$ and $q_k$ are the matrix and vector defining the rotated polytope.
For phase $k$, the relative foot position constraints can be formulated as follows:
\begin{equation}
    %\mathcal{Q}_{k-1}^j (p_k - \mbox{p}_{k-1}) \leq q_{k-1}^j + M a_{k-1}^j \;\;\;\; \forall j
    \mathcal{Q}_{k-1} (\mbox{p}_k - \mbox{p}_{k-1}) \leq q_{k-1}
\end{equation}

\section{Complete MIP Formulation \label{appendix:complete_mip_formulation}}
The complete formulation of the contact-before-motion MIP contact planning problem with the center of mass positions and the feasibility constraints of previous Appendix \ref{appendix:feasibility_constr} is:

\begin{align}
    \textrm{\textbf{find}}  \quad & \mbox{P}=[\mbox{p}_1,...,\mbox{p}_n], \; \mbox{p}_i \in \mathbb{R}^{3 \times n} \nonumber\\
                            \quad & R=[\mbox{r}_1,...,\mbox{r}_n], \; r_i \in \mathbb{R}^{3 \times n} \nonumber\\
                            \quad & \textcolor{blue}{\mbox{C}=[\mbox{c}_{0,1},\mbox{c}_{1,0},\mbox{c}_{1,1},...,\mbox{c}_{n,0},\mbox{c}_{n,1}], \; \mbox{c}_{k,e} \in \mathbb{R}^3} \nonumber\\
                            \quad & A=[\mbox{a}_1,...,\mbox{a}_n], \; \mbox{a}_i \in \{0,1\}^{m} \nonumber\\
                            \quad & \beta=[\beta_1,...,\beta_n], \; \beta_i \in \mathbb{R}^{m} \nonumber\\
                            %\quad & \alpha=[\alpha_1,...,\alpha_n] \in \mathbb{R}_{+}^{n}\\
    %\min_{w,b,\xi} \quad & \frac{1}{2}w^{t}w+C\sum_{i=1}^{N}{\xi_{i}}\\
    \textrm{\textbf{min}}  \quad & l(\mbox{P},\mbox{R},\textcolor{blue}{\mbox{C}}) \nonumber\\
    \textrm{\textbf{s.t.}}  \quad & \{\mbox{P},\mbox{R},\textcolor{blue}{\mbox{C}}\} \in \mathcal{I} \cap \mathcal{G} \nonumber\\
                            \quad & \forall i \in \{1,..,n\} : \nonumber\\
                                \quad & \quad \mbox{card}(a_i) = m-1  \nonumber\\
                                \quad & \quad \forall j \in \{1,..,m\} : \nonumber\\
                                    \quad & \quad \quad S^j \mbox{p}_i \leq s^j + M a_i^j \textbf{1} \nonumber\\
                                    \quad & \quad \quad (p_i)^{\intercal} \textbf{d}^j = e^j + \beta_i^j \nonumber\\
                                    \quad & \quad \quad ||\beta_i^j||_1 \leq M a_i^j \nonumber\\
                                    %\quad & \quad \quad \textcolor{blue}{F^{j}_{i-1} (\mbox{c}_{i,\:0}-\mbox{p}_{i-1}) \leq  f_{i-1}^{j} + M a_{i-1}^{j} } \nonumber\\
                                    %\quad & \quad \quad \textcolor{blue}{F^{j}_{i} (\mbox{c}_{i,\:1}-\mbox{p}_{i}) \leq  f^j_{i} + M a{i}^{j} }  \nonumber\\
                                    %\quad & \quad \quad \textcolor{blue}{R_{i}^j (\mbox{c}_{i,e} - \mbox{p}_{i}) \leq r_i^j + M a_i^j }  \nonumber\\
                                    %\quad & \quad \quad \textcolor{blue}{R_{i-1}^j (\mbox{c}_{i-1,e} - \mbox{p}_{i-1}) \leq r_{i-1}^j + M a_{i-1}^j } \nonumber\\
                                    %\quad & \quad \quad \textcolor{blue}{\mathcal{Q}_{i-1}^j (p_i - \mbox{p}_{i-1}) \leq q_{i-1}^j + M a_{i-1}^j} \nonumber
                                    \quad & \quad \quad \textcolor{blue}{F_{i-1} (\mbox{c}_{i,\:0}-\mbox{p}_{i-1}) \leq  f_{i-1}} \nonumber\\
                                    \quad & \quad \quad \textcolor{blue}{F_{i} (\mbox{c}_{i,\:1}-\mbox{p}_{i}) \leq  f_{i}}  \nonumber\\
                                    \quad & \quad \quad \textcolor{blue}{R_{i} (\mbox{c}_{i,e} - \mbox{p}_{i}) \leq r_i}  \nonumber\\
                                    \quad & \quad \quad \textcolor{blue}{R_{i-1} (\mbox{c}_{i-1,e} - \mbox{p}_{i-1}) \leq r_{i-1}} \nonumber\\
                                    \quad & \quad \quad \textcolor{blue}{\mathcal{Q}_{i-1} (p_i - \mbox{p}_{i-1}) \leq q_{i-1}} \nonumber
\end{align}